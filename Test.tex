\documentclass[12pt,a4paper]{article}
\usepackage[utf8]{inputenc}
\usepackage[T1]{fontenc}
\usepackage{wrapfig}
\usepackage[french]{babel}
\usepackage[pdftex]{graphicx}
\usepackage{geometry}
\usepackage{hyperref}
\geometry{hmargin=2.5cm,vmargin=1.5cm}
\usepackage{amsmath,amsfonts,amssymb}
\setlength{\parindent}{0cm}
\setlength{\parskip}{1ex plus 0.5ex minus 0.2ex}
\newcommand{\hsp}{\hspace{20pt}}
\newcommand{\HRule}{\rule{\linewidth}{0.5mm}}


\title{Les métamatériaux}
\author{Aldegheri Emeline Collard Maxence}
\date{2020}
\begin{document}
\begin{titlepage}
  \begin{sffamily}
  \begin{center}

    % Upper part of the page. The '~' is needed because \\
    % only works if a paragraph has started.
	\includegraphics[scale=0.20]{../Documents/ShareX/Screenshots/2020-03/108320213_o.png}~\\[1.0cm] 
    \textsc{\LARGE Faculté Jean Perrin}\\[1.5cm]
	
    \textsc{\Large Recherche Bibliographique}\\[1.0cm]
    % Title
    \HRule \\[0.4cm]
    { \huge \bfseries Les métamatériaux\\[0.4cm] }
    \HRule \\[1cm]

	\includegraphics[scale=0.25]{../Documents/ShareX/Screenshots/2020-03/chrome_chkA9LCw4W.png}~\\[1.0cm] 
    % Author and supervisor
    \begin{minipage}{0.4\textwidth}
      \begin{flushleft} \large
        \textsc{Aldegheri} Emeline\\
        \textsc{Collard} Maxence \\
      \end{flushleft}
    \end{minipage}
    \begin{minipage}{0.4\textwidth}
      \begin{flushright} \large
        \emph{Module :} Initiation à la recherche\\
        \emph{Enseignant : } M. Blach
      \end{flushright}
    \end{minipage}

    \vfill

    % Bottom of the page
    {\large  2020}

  \end{center}
  \end{sffamily}
\end{titlepage}
\tableofcontents
\newpage
\section{Introduction}

\setlength{\parindent}{1cm} Pour commencer, il faut définir ce qu’est un métamatériaux. Tout d’abord, le nom métamatériaux est composé de deux mots :
\begin{itemize}
\item Le préfix méta venant du grec $“\mu\epsilon\tau\alpha”$ , exprimant le fait d’aller au delà 
\item Le nom matériaux, le pluriel de matériel venant du latin materia, exprimant la matière
\end{itemize} 
\begin{wrapfigure}{l}{0.5\textwidth}\centering
 \includegraphics[scale=0.3]{../Documents/ShareX/Screenshots/2020-03/12388_1.jpg}
 \caption{Coupe de Lycurgue (à gauche) éclairée de face; (à droite) éclairée de l'intérieur. }
\end{wrapfigure}
  
La première apparition du nom métamatériaux est daté de 1999, et a été nommé par Rodger M.Walser, chercheur à l’université du Texas, à Austin. Il définit ainsi les métamatériaux comme : “ Macroscopic composites having a synthetic, three-dimensional, periodic cellular architecture designed to produce an optimized combination, not available in nature, of two or more responses” \cite{Fan2017}, c’est à dire comme un composite macroscopique ayant une architecture synthétique périodique tridimensionnelle, conçu pour produire des propriétés et dont la matière n’existe pas dans la nature. En réalité, la fabrication de métamatériaux est possiblement daté du 4 ème siècle, durant l’époque de l’empire Romain. Les propriétés de ces métamatériaux n’étaient pas réellement compris à l’époque. Un exemple de ces métamatériaux datant de l’époque romaine est la coupe de lycurgue, dont la propriété était de changer de couleur selon l’exposition de la lumière.\cite{Thakur}\\

 La première fabrication “voulue” date des années 1940, fabriqué par Winston .E. Kock, ingénieur électrique et chercheur. Il fut le premier directeur du centre électronique de recherche (Electronics Research Center - ERC ) à la NASA, à Cambridge, Massachusetts. Il nota dans son livre "sound Waves \& Light Waves the Fundamenta"\cite{Kock} , daté de 1965, des comportements de structures artificielles ainsi que ces structures, qui seront plus tard nommés métamatériaux.\\

	Un second article évoquant les propriétés des métamatériaux fut publié en 1969 dans l’article “SERGEĬ MIKHAĬLOVICH RYTOV" (in Honor of his 60th Birthday)” \cite{Veselago}, et fut introduit par Victor Veselago, un physicien théoricien et professeur à l’institut physique et de technologie de moscou.  \\
	
	Dans cet article, Victor Veselago introduit les possibles propriétés des métamatériaux, tel que l’indice de réfraction negatif. Son objet principal d’étude dans cet article est la propagation de l’onde plane dans un matériau dont la permittivité et la perméabilité ont été supposé négative. Il montra que théoriquement, une onde plane monochromatique ayant comme direction le vecteur de Poynting qui est anti parallèle à la direction de la vitesse de phase, qui est différent de la propagation d’onde plane dans un milieu classique.
\begin{figure}[hbtp]
	\centering
	\includegraphics[scale=0.32]{../Documents/ShareX/Screenshots/2020-03/mp.png}
	\caption{Shéma illustrant le trièdre formé par l'excitation magnétique, le champ électrique et le vecteur de Poynting}
	\end{figure} \\ 
	Ce fut seulement en 1999 que le nom matériaux apparu comme décrit précédemment ( Cf ligne 4).\\
	
	L’attrait pour les métamatériaux pris 30 ans avant que la communauté scientifique s'intéresse à ce domaine. Cela est due au manque d’expérience concluante.
La première expérience concluante fut publié en 1999 par John Pendry ainsi que se adjoints, dans laquelle il montra qu’ils avaient réussi à créer un matériaux dont la permittivité était négative. Dans cette expérience, ils utilisaient le plasma pour montrer une permittivité négative. Ce matériaux est conçu de fil électrique en anneau afin d’avoir des “Split-ring resonator”\\

	John Pendry est très connu pour une de ses idées, celle de la possibilité de concevoir une “cape d’invisibilité”, qui fut publié dans un document avec David R. Smith, nommé “ Theoretical Blueprint For Invisibility Cloak Reported“\cite{Duke} daté de 2006.

\section{Définition}
	Pour commencer, en quoi ces matériaux possèdent des propriétés électromagnétiques impossible à retrouver dans la nature ? Pourquoi sont ils dit de “main gauche” ? Quelles sont les propriétés qu’il en découle ?\\
	\subsection{Indice de réfraction}
	Pour débuter, pourquoi dit-on que ces matériaux ont des propriétés impossible à retrouver dans la nature. Tout d’abord la propriétés la plus compréhensible est que son indice de réfraction est négatif, ce qui peut être représenté par un schéma : \\
\begin{wrapfigure}{l}{0.5\textwidth}\centering
 \includegraphics[scale=0.6]{../Documents/ShareX/Screenshots/2020-03/Exemple_de_Réfraction_négative.png}
 \caption{Schéma Illustrant la réfraction d'un faisceau selon l'indice de réfraction}
\end{wrapfigure}
Cette propriété permet donc lorsqu'une onde plane traverse un matériau dit de “main-gauche” (MMG) ou métamatériaux, son onde transmise est à l’opposé de celle transmise dans un matériau conventionnel. Elle forme donc un angle avec la normal qui est égal à l’opposé de l’angle formé entre l’onde transmise dans un matériau conventionnel et la normal.\cite{Fan2017} \\  \\ \\ \\ \\
	\subsection{Permittivité et perméabilité}
Ensuite, on peut définir l’indice de réfraction par la relation :\hspace{1.0cm} \large \textbf{$n=\frac{c}{v}$} \\

\normalsize Or dans le vide, on a : \hspace{1.0cm} \large \textbf{$c=\frac{1}{\sqrt{\varepsilon_{0} . \mu_{0}}}$} \\
\normalsize Et également, dans un milieu diélectrique homogène, isotrope et transparent : \large \textbf{$v=\frac{1}{\sqrt{\varepsilon . \mu}}$}\\

\normalsize On retrouve donc pour l'indice de réfraction :\hspace{0.5cm} \large \textbf{$n=\sqrt{\frac{\varepsilon . \mu}{\varepsilon_{0} . \mu_{0}}}=\sqrt{\varepsilon-{r} . \mu_{r}}$}\\ \normalsize

On peut donc voir que l'indice de réfraction est relié à la permittivité et perméabilité du milieu. Or plusieurs cas peuvent se présenter : \\
\begin{itemize}
\item Les matériaux courant où la permittivité est positive et la perméabilité est positive
\item Les Ferrites où la permittivité est positive et la perméabilité est négative
\item Le Plasma où la permittivité est négative et la perméabilité est positive
\item Les métamatériaux où la permittivité est négative et la perméabilité est négative
\end{itemize} 
	Ainsi, on peut donc donc extraire deux nouvelles propriétés des métamatériaux, qui sont que sa permittivité et perméabilité est négative.\cite{Fan2017}
	\subsection{Trièdre formé}
	Par ailleurs, le vecteur d’onde k, le champ magnétique et le champ électrique forme un trièdre. Dans le cas des métamatériaux, ce trièdre formé est inversé , car le vecteur d’onde k défini par : \hspace{4cm} \large \textbf{$\overrightarrow{k}=\frac{n\omega}{c}$}\normalsize \\ \\
	Dans le cas des métamatériaux, l’indice de réfraction est négatif, alors nous avons un vecteur k opposé :\cite{DeLustrac} \\
\begin{wrapfigure}{l}{1\textwidth}
	\centering
	\includegraphics[scale=1]{../Documents/ShareX/Screenshots/2020-03/chrome_Z1FtSXs0vP.png}
	\caption{Schéma illustrant le triplet E, H et k selon la règle de la main droite pour (a) et la règle de la main gauche pour (b).}
\end{wrapfigure}\\ \\ \\ \\ \\ \\ \\ \\ \\ \\ \\ \\ \\ \\
	\subsection{Vitesse de phase et de groupe}
	De plus, pour les MMG, les vitesses de phases et de groupes sont opposés, alors que pour un milieu classique, ces vitesses sont de même sens. Cela s’explique par leur calcul théorique.\\ Pour les ondes électromagnétiques, on a :\\
\begin{center}
\large \textbf{$n=\frac{c}{v_{\Phi}}$} \hspace{4cm} \textbf{$v_{g}=\frac{d\omega}{dk}$}\normalsize
\end{center} 
Ainsi, on remarque donc que la vitesse de phase est négative car l'indice de réfraction est négatif à l'inverse de la vitesse de groupe qui est positive, donc ces deux vitesses sont opposées. \cite{Raffy}
	\subsection{Effet Doppler}
	L’effet Doppler est aussi inversé pour le cas des MMG, on a la formule donnant la différence de fréquence entre celle reçu et celle émise par:\hspace{0.5cm} \large \textbf{$\Delta\omega=\omega_{source}-\omega_{recepteur}$}\normalsize \hspace{0.2cm}
	où $\omega$ est la fréquence\\ 
	
	De plus, la fréquence peut s’exprimer en fonction de la fréquence émise, de la vitesse de déplacement et de la vitesse de l’onde (vp) par : \hspace{0.5cm} \large \textbf{$\Delta\omega = \pm\omega_{0}\frac{v}{v_{p}}$}\normalsize \\
	Le signe $\pm$ est appliqué pour correspondre aux milieux ordinaires et MMG.\\
	
	Ainsi par la relation liant la vitesse v à la célérité et à l’indice de réfraction on peut donc écrire :\hspace{5cm} \large \textbf{$\Delta\omega=\omega_{0}\frac{nv}{c}$} \normalsize \\
	Or dans un MMG, $n<0$ alors l’effet Doppler est inversé.\cite{Lee}\\ \\ \\
	
\section{Champs d'application}
	\subsection{Dans l'optique}
	Une des applications les plus connues de cette technologie est celle de la lentille parfaite aussi appelée super lentille. Jusqu’en 2006, les métamatériaux ont d’abord été étudié pour l’application dans le domaine des micro-ondes. En effet, les chercheurs ne pensaient pas que la limite de diffraction pouvait être dépassée grâce à ce matériau. Cette limite certifiée qu’il était impossible de voir un objet s’il était plus petit que la longueur d’onde de l’onde émise mais grâce à l’indice de réfraction négatif des métamatériaux, il est possible d’aller au dessus de cette limite.\newpage
\begin{wrapfigure}{l}{0.5\textwidth}\centering
	\includegraphics[scale=0.48]{../Documents/ShareX/Screenshots/2020-03/nano.jpg}
	\caption{Résultats de l'expérience de l'équipe du Professeur Xiang Zhang.}
\end{wrapfigure}
C’est grâce à deux équipes américaines que l’on voit naître les premières super lentilles. Tout d’abord, le professeur Xiang Zhang, de l’Université de Californie à Berkeley, réalise une lentille de la forme d’un demi-cylindre pavé, mesurant 35 nanomètres, formé d’oxyde d’argent et d’aluminium. Pour montrer l’efficacité de la lentille, l’équipe du professeur Xiang Zhang observe un objet, un mot «NANO» de l’ordre de la dizaine de nanomètre,  à l’aide d’une sonde ionique focalisée et de deux sortes de lentilles, une super lentille (figure b) et une lentille de microscope standard (figure d ). \\
\begin{wrapfigure}{r}{5cm}
  \centering
\includegraphics[scale=0.6]{../Documents/ShareX/Screenshots/2020-03/ht.jpg}
\caption{Photo de la constitution de la lentille de l'équipe du Docteur Smolyaninov}
\end{wrapfigure}

	On s’aperçoit que la ligne du A est beaucoup plus épaisse lorsque l’on utilise une lentille standard, 380 nm alors qu’elle est de 90nm pour la super lentille. Simultanément, l’équipe du docteur Igor Smolyaninov à l’Université de Maryland, ont réalisé une lentille grâce à des cercles concentriques de polyméthyl méthacrylate (PMMA) sur un film d’or. Grâce à l’éclairage d’un rayon laser, une vague d’électron en mouvement, aussi appelé un plasmon, est crée et il se propage à l’interface or-PMMA. Cela a permit d’avoir un indice de réfraction négatif, principal propriété d’un méta- matériau et de mesuré une résolution de 70 nanomètres. \cite{Goudet2007}  \\ 

	Ces premières expériences ont pu montré qu’il était possible de créer une lentille qui fonctionnait à des longueurs d’ondes faibles et qui défiait la limite de diffraction, c’est-à-dire que l’on peut voir un objet qui est plus petit que la lumière d’onde émise. Pour réaliser une lentille parfaite, la méthode de lithographie est utilisée. Cette dernière a été inventée en 1796 par l’imprimeur allemand Aloys Senefelder, elle permet de reproduire sur un pierre calcaire un texte écrit à l’encre ou au crayon. Elle apporte aussi une précision dans la formation d’une lentille, ce qui est très important car les capacités de la lentille sont basées sur sa finesse. En effet, la lentille reconstruit l’image plus nette plus elle est fine. \\

    La super lentille est une avancée technologique avec quelques avantages comme sa facilité de production grâce à la méthode de lithographie vu précédemment mais aussi elle est très légère et peu encombrante. Cette lentille est donc une révolution pour certains domaines. Dans celui du médical, il est possible de créer un microscope optique capable de voir de nos propres yeux des structures minuscules, comme par exemple de l’ADN, contrairement à la méthode actuelle qui est de visionner par microscope électronique. Mais aussi dans le domaine des jeux vidéos comme pour la création d’un autre type de casque de réalité virtuelle.\\
	\subsection{Dans le génie civil}

\begin{wrapfigure}{l}{7cm}]
\centering
\includegraphics[scale=0.22]{../Documents/ShareX/Screenshots/2020-03/838_tres_violent_seisme_au_large_8780_hd.jpg}
\caption{Carte du Japon situant l'épicentre du séisme de 2011}
\end{wrapfigure}

	
	Le 11 mars 2011, un séisme apparu à 9.1km au large des côtes Pacifique du Japon, de magnitude 7.2. Ce séisme créa un tsunami de 39 mètres au point le plus haut, soit une vague aussi haute qu’un immeuble de 9 étages. Le tsunami frappa les côtes japonaises à différents endroits dont la centrale nucléaire de Fukishima. Cette catastrophe naturel entraîna de nombreux morts, environ 15 897 morts et également une catastrophe nucléaire. 
Par ailleurs, les séismes entraînent de nombreuses victimes, on en dénombra 830 000 en 2010, on peut donc se poser de l'efficacité des systèmes parasismique actuels.
\begin{wrapfigure}{r}{5cm}
\centering
\includegraphics[scale=0.25]{../Documents/ShareX/Screenshots/2020-03/canvas.png}
\caption{Schéma illustrant une installation de protection contre les séismes }
\end{wrapfigure}

Les métamatériaux sont peut-être une solution efficace contre la lutte et la réduction des crises liées aux séismes et tsunamis. Dans ce contexte, les métamatériaux permettraient de créer des systèmes de protections sismique et de protection de la houle. \\
 Les propriétés des métamatériaux pourraient théoriquement dévier l’onde de la zone sensible à protéger ( habitations, infrastructures dangereuses, … ). \\

A l’heure actuelle, la protection majeur contre les événements sismiques est liée aux processus de fabrication des bâtiments. Cela implique une étude géotechnique des sols et des soubassements, c’est à dire que l’on évitera les reliefs, les zones limites entre sols mous et rocheux. Un traitement de sols et un terrassement particulier autour du bâtiment avec des matériaux de meilleur compacité. Des fondations découplées avec un système d’isolation, comme des galets en caoutchouc, ainsi qu’un vide sanitaire permettant l’isolation des remontées d’humidité et de protection sismique. \\



	Les techniques actuelles sont donc basées sur la construction, ainsi des immeubles anciens ne pourraient pas être aussi bien protégés que des immeubles parasismiques. A l’inverse des techniques actuelles, les métamatériaux ne nécessitent pas de modification sur les bâtiments, ils nécessitent l’installation d’un système aux abords de la zone à protéger.\newpage
\begin{wrapfigure}{l}{7.0cm}]
\centering
\includegraphics[scale=0.40]{../Documents/ShareX/Screenshots/2020-03/6742b695ef_50090412_snapshot-of-earthquake-like-crash-testing.jpg}
\caption{Image d'un test effectué entre un batiment courant et un batiment parasismique en taille réduite}
\end{wrapfigure}

	Pour cela, l’objectif principal était de dévier les ondes sismiques et plus particulièrement les ondes Rayleigh qui sont les plus destructives. Le principe est simple et il est comparable dans le domaine acoustique.\\

	Afin d’expliquer les techniques utilisées, je vais m’appuyer sur un document datant du avril 2014, nommé “ Experiments on Seismic Metamaterials: Molding Surface Waves “, produit par S. Brûlé, E. H. Javelaud, S. Enoch, et S. Guennea.\cite{1} \\

	Le but des expériences menées était de montrer qu’il était possible de créer une “cape d’invisibilité” par le biai de trou à différents endroits en jouant sur la densité volumique des matériaux. \\

	Une étude théorique a tout d’abord été effectué en partant des équations de Navier-Stokes tri-dimensionnels. Pour simplifier ces équations, l’étude sera selon le modèle asymptotique qui permet de calculer uniquement les ondes à l’interface air-sol. \\

	Une seconde approximation fut effectuée, c’est l’approximation Mindlin sur la théorie des plaques (Permet de calculer les déformations et les contraintes dans une plaque qui est soumises à des charges). \\ 

	Les équations de Navier-Stokes deviennent donc : \\

\large \hspace{3cm} \textbf{kh($\nabla\cdot\mu\nabla\Psi+ \mu \frac{\partial u}{\partial y}+ \frac{\partial \upsilon}{\partial y}) + \rho h\omega^{2}\Psi=0 $}\normalsize\\\\
\large \hspace{3cm} \textbf{$\nabla\cdot D(1-\upsilon)\nabla u+\frac{\partial}{\partial x}[D(1+\upsilon)(\frac{\partial u}{\partial x}+\frac{\partial \upsilon}{\partial y})]-2k\mu h(\frac{\partial\Psi}{\partial x}+u)+\frac{\rho h^{3}}{6}\omega^{2}u=0$}\normalsize\\\\
\large \hspace{3cm} \textbf{$\nabla\cdot D(1-\upsilon)\nabla v+\frac{\partial}{\partial y}[D(1+\upsilon)(\frac{\partial u}{\partial x}+\frac{\partial \upsilon}{\partial y})]-2k\mu h(\frac{\partial\Psi}{\partial y}+\upsilon)+\frac{\rho h^{3}}{6}\omega^{2}\upsilon=0$}\normalsize\\

	où k est la constante de normalisation de Mindlin, $\mu$ est le module de cisaillement ce qui correspond à une grandeur physique intrinsèque intervenant à la caractérisation des déformations liées à des efforts de cisaillement, h l’épaisseur de la plaque, $\rho$ la densité volumique , $\upsilon$ le coefficient de Poisson qui permet de caractériser la contraction de la matière, D la rigidité de la plaque, x;y;z sont les coordonnées dans l’espace, t le temps, $\omega$ la pulsation de flexion de l’onde et on peut définir le module de cisaillement ou second coefficient de Lalé comme : \hspace{2cm} \large \textbf{$\mu =\frac{E}{[2(1+\upsilon)]}$}\normalsize \\
	Ou E est le module de Young qui est la constante reliant contrainte de traction et déformation.\\
	
	Une équation différentielle partielle d'ordre 4 peut être dérivée pour $\Psi$, définissant l’amplitude de déplacement le long de l’axe z, en éliminant u et $\omega$. \\ 
	
	Lorsque les termes d’inertie et de déformations/cisaillement sont omis, cela amène aux approximations de Kirchhoff-Love pour les plaques minces, approximation qui montre que les plaques conservent leurs formes : \\
	
	\large\hspace{3cm}\textbf{$\rho^{-1}\nabla\cdot[E^{\frac{-1}{2}}\nabla\rho^{-1}\nabla\cdot(E^{\frac{1}{2}}\nabla\Psi)-\omega^{2}\frac{\rho h}{D}\Psi =0]$} \normalsize\\
	
	Pour le calcul théorique, on aura pris un sol ayant une masse volumique de 1500$kg/m^{3}$, des trous vides espacés de 1.73 mètres de 0.32 mètres de diamètre, de 5 mètres de profondeur(=plaques de 5 mètres) et placés en lignes. De plus, après un relevé sur le terrain, on a E = 100 MPa ( MégaPascale) et $\upsilon$ = 0.3.\\

	En utilisant la dernière équation et en entrant toutes ces conditions dans un logiciel de simulation (COMSOL), un graphique montrant la dispersion des ondes en fonction du nombre d’onde et de la fréquence.\\
	
\begin{figure}[hbtp]
	\centering
	\includegraphics[scale=0.8]{../Documents/ShareX/Screenshots/2020-03/chrome_MgUJD1op0H.png}
	\caption{Simulation des coubes de réflexion en fonction du nombre d'onde et de la fréquence}
\end{figure} 
	En observant ce graphique, on remarque que ce modèle suggère un réfraction négative sur le domaine de 3 à 10 Hz ( Pente négative). Ainsi, en effectuant le même test en pratique dans la zone de réfraction négative, nous devrions voir apparaître un espace cachée des ondes sismiques, ce qu’il fut fait à Grenoble en août 2012 en collaboration avec le groupe Menard, spécialiste dans l’amélioration des sols.\\

	Le principe est simple, une source émettrice, des capteurs à différents endroits et des trous placées à des endroits déterminées.\\
	
\begin{figure}[hbtp]
	\centering
	\includegraphics[scale=0.45]{../Documents/ShareX/Screenshots/2020-03/chrome_F2oqxtwNOp.png}
	\caption{Schéma du protocole expérimentale effectué à Grenoble en août 2012}
\end{figure}
\begin{figure}[h]
	\centering
	\includegraphics[scale=0.5]{../Documents/ShareX/Screenshots/2020-03/chrome_8Gk62pOVNL.png}
	\caption{Photographie du test mis en place à Grenoble en août 2012}
\end{figure} 
	Dans cette première expérience, on repris les mêmes conditions que dans le modèle théorique et en prenant une fréquence d’émission de 50 Hz.
Les capteurs sont placés dans la zone verte afin de mesurer l’onde produite
Une cartographie (a) de l'énergie enregistrée moins l'énergie émise en fonction de la position fut effectué. Une seconde (b) fut effectué où l'énergie reçu divisé par l'énergie initial.\\
\begin{figure}[hbtp]
	\centering
	\includegraphics[scale=0.5]{../Documents/ShareX/Screenshots/2020-03/JEEkxp5KMc.png}
	\caption{Cartographie représentant l'energie J2-J1 (a) et J2/J1 en fonction de la position}
\end{figure}\\
	Les zones en bleu sont les zones où l'énergie sismique est très faible voir nul pour le graphique (b).
On peut donc remarquer qu’il y a en effet des zones “d’ombres” ou l’onde sismique est peu ressenti ce qui semble prometteur.
Malgré cela, on remarque qu’il y a beaucoup de bruit et ce modèle n’est pas encore parfait, ce qui peut probablement être dû au protocole, à l’incertitude de mesure, … \\

	C’est pourquoi un second test a été effectué dans une seconde zone géographique, à Saint-Priest le 26 septembre 2012, toujours avec l’entreprise Ménard mais en utilisant une configuration différente : \\
	\begin{itemize}
		\item Des trou de 3 m de diamètre et de profondeur 5mètres
		\item Des capteurs espacés de 5mètres
		\item Une fréquence d’émission de 5 Hz
	\end{itemize} 
	
\begin{figure}[hbtp]
\centering
\includegraphics[scale=0.8]{../Documents/ShareX/Screenshots/2020-03/chrome_xv08uM3wo7.png}
\caption{Photographie du nouveau test effectué à Saint-Priest le 26 septembre 2012}
\end{figure} 
	Deux test ont été effectués, un premier ou l’épicentre se trouve à une distance d et un second à une distance d/2. Pour ces deux test, une masse de 17 tonnes suspendu à une grue fut lâchée à 20 m de hauteur, ce qui correspond à un microséisme de niveau 4 sur l’échelle de Richter.
	Deux cartographie on été effectué ou affiche l'énergie capté en fonction de la position et que l’on compare au modèle théorique calculé au préalable :\\
	
\begin{figure}[hbtp]
\centering
\includegraphics[scale=0.5]{../Documents/ShareX/Screenshots/2020-03/pd2bqUFiEM.png}
\caption{Cartographie expérimentales et théorique de l'energie perçue en fonction de la position et de la distance d'impact}
\end{figure}
		\newpage On voit clairement que les résultats théorique collent fortement avec les simulations ce qui confirme et permettent d’affiner les calculs. 
Ainsi, on peut dire clairement que les métamatériaux ont un fort potentiel d’application dans le domaine du génie civil, dans la protection sismique mais on peut également appliquer ceci à la protection côtière, pour la protection de tsunami ou également de la houle dans des zones à risques.\cite{2} \\ \\

	Des expériences pour la protection des côtes sont en cours dans le canal à houle de 17m de Central Marseille.\cite{3}\\
	
	Cet aspect des métamatériaux est plutôt encourageant pour le futur et pour le développement de nouvelles technologie et techniques de construction pour le génie civil .

\subsection{Autres domaines}
	De plus, les métamatériaux peuvent avoir une application direct dans des domaines différents à ceux énnoncés ci-dessus. \\
	Ils peuvent entrer dans la miniaturisation d'antenne large bande comme il est en partie traité dans la thèse de N.Kristou\cite{4}
	
	
	
\section{Conclusion}
	
\newpage \bibliographystyle{abbrv}
\bibliography{bibli}
\listoffigures


	


	
	





 
\end{document}